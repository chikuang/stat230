% Options for packages loaded elsewhere
\PassOptionsToPackage{unicode}{hyperref}
\PassOptionsToPackage{hyphens}{url}
%
\documentclass[
]{book}
\usepackage{amsmath,amssymb}
\usepackage{iftex}
\ifPDFTeX
  \usepackage[T1]{fontenc}
  \usepackage[utf8]{inputenc}
  \usepackage{textcomp} % provide euro and other symbols
\else % if luatex or xetex
  \usepackage{unicode-math} % this also loads fontspec
  \defaultfontfeatures{Scale=MatchLowercase}
  \defaultfontfeatures[\rmfamily]{Ligatures=TeX,Scale=1}
\fi
\usepackage{lmodern}
\ifPDFTeX\else
  % xetex/luatex font selection
\fi
% Use upquote if available, for straight quotes in verbatim environments
\IfFileExists{upquote.sty}{\usepackage{upquote}}{}
\IfFileExists{microtype.sty}{% use microtype if available
  \usepackage[]{microtype}
  \UseMicrotypeSet[protrusion]{basicmath} % disable protrusion for tt fonts
}{}
\makeatletter
\@ifundefined{KOMAClassName}{% if non-KOMA class
  \IfFileExists{parskip.sty}{%
    \usepackage{parskip}
  }{% else
    \setlength{\parindent}{0pt}
    \setlength{\parskip}{6pt plus 2pt minus 1pt}}
}{% if KOMA class
  \KOMAoptions{parskip=half}}
\makeatother
\usepackage{xcolor}
\usepackage{color}
\usepackage{fancyvrb}
\newcommand{\VerbBar}{|}
\newcommand{\VERB}{\Verb[commandchars=\\\{\}]}
\DefineVerbatimEnvironment{Highlighting}{Verbatim}{commandchars=\\\{\}}
% Add ',fontsize=\small' for more characters per line
\usepackage{framed}
\definecolor{shadecolor}{RGB}{248,248,248}
\newenvironment{Shaded}{\begin{snugshade}}{\end{snugshade}}
\newcommand{\AlertTok}[1]{\textcolor[rgb]{0.94,0.16,0.16}{#1}}
\newcommand{\AnnotationTok}[1]{\textcolor[rgb]{0.56,0.35,0.01}{\textbf{\textit{#1}}}}
\newcommand{\AttributeTok}[1]{\textcolor[rgb]{0.13,0.29,0.53}{#1}}
\newcommand{\BaseNTok}[1]{\textcolor[rgb]{0.00,0.00,0.81}{#1}}
\newcommand{\BuiltInTok}[1]{#1}
\newcommand{\CharTok}[1]{\textcolor[rgb]{0.31,0.60,0.02}{#1}}
\newcommand{\CommentTok}[1]{\textcolor[rgb]{0.56,0.35,0.01}{\textit{#1}}}
\newcommand{\CommentVarTok}[1]{\textcolor[rgb]{0.56,0.35,0.01}{\textbf{\textit{#1}}}}
\newcommand{\ConstantTok}[1]{\textcolor[rgb]{0.56,0.35,0.01}{#1}}
\newcommand{\ControlFlowTok}[1]{\textcolor[rgb]{0.13,0.29,0.53}{\textbf{#1}}}
\newcommand{\DataTypeTok}[1]{\textcolor[rgb]{0.13,0.29,0.53}{#1}}
\newcommand{\DecValTok}[1]{\textcolor[rgb]{0.00,0.00,0.81}{#1}}
\newcommand{\DocumentationTok}[1]{\textcolor[rgb]{0.56,0.35,0.01}{\textbf{\textit{#1}}}}
\newcommand{\ErrorTok}[1]{\textcolor[rgb]{0.64,0.00,0.00}{\textbf{#1}}}
\newcommand{\ExtensionTok}[1]{#1}
\newcommand{\FloatTok}[1]{\textcolor[rgb]{0.00,0.00,0.81}{#1}}
\newcommand{\FunctionTok}[1]{\textcolor[rgb]{0.13,0.29,0.53}{\textbf{#1}}}
\newcommand{\ImportTok}[1]{#1}
\newcommand{\InformationTok}[1]{\textcolor[rgb]{0.56,0.35,0.01}{\textbf{\textit{#1}}}}
\newcommand{\KeywordTok}[1]{\textcolor[rgb]{0.13,0.29,0.53}{\textbf{#1}}}
\newcommand{\NormalTok}[1]{#1}
\newcommand{\OperatorTok}[1]{\textcolor[rgb]{0.81,0.36,0.00}{\textbf{#1}}}
\newcommand{\OtherTok}[1]{\textcolor[rgb]{0.56,0.35,0.01}{#1}}
\newcommand{\PreprocessorTok}[1]{\textcolor[rgb]{0.56,0.35,0.01}{\textit{#1}}}
\newcommand{\RegionMarkerTok}[1]{#1}
\newcommand{\SpecialCharTok}[1]{\textcolor[rgb]{0.81,0.36,0.00}{\textbf{#1}}}
\newcommand{\SpecialStringTok}[1]{\textcolor[rgb]{0.31,0.60,0.02}{#1}}
\newcommand{\StringTok}[1]{\textcolor[rgb]{0.31,0.60,0.02}{#1}}
\newcommand{\VariableTok}[1]{\textcolor[rgb]{0.00,0.00,0.00}{#1}}
\newcommand{\VerbatimStringTok}[1]{\textcolor[rgb]{0.31,0.60,0.02}{#1}}
\newcommand{\WarningTok}[1]{\textcolor[rgb]{0.56,0.35,0.01}{\textbf{\textit{#1}}}}
\usepackage{longtable,booktabs,array}
\usepackage{calc} % for calculating minipage widths
% Correct order of tables after \paragraph or \subparagraph
\usepackage{etoolbox}
\makeatletter
\patchcmd\longtable{\par}{\if@noskipsec\mbox{}\fi\par}{}{}
\makeatother
% Allow footnotes in longtable head/foot
\IfFileExists{footnotehyper.sty}{\usepackage{footnotehyper}}{\usepackage{footnote}}
\makesavenoteenv{longtable}
\usepackage{graphicx}
\makeatletter
\def\maxwidth{\ifdim\Gin@nat@width>\linewidth\linewidth\else\Gin@nat@width\fi}
\def\maxheight{\ifdim\Gin@nat@height>\textheight\textheight\else\Gin@nat@height\fi}
\makeatother
% Scale images if necessary, so that they will not overflow the page
% margins by default, and it is still possible to overwrite the defaults
% using explicit options in \includegraphics[width, height, ...]{}
\setkeys{Gin}{width=\maxwidth,height=\maxheight,keepaspectratio}
% Set default figure placement to htbp
\makeatletter
\def\fps@figure{htbp}
\makeatother
\setlength{\emergencystretch}{3em} % prevent overfull lines
\providecommand{\tightlist}{%
  \setlength{\itemsep}{0pt}\setlength{\parskip}{0pt}}
\setcounter{secnumdepth}{5}
\usepackage{booktabs}
\ifLuaTeX
  \usepackage{selnolig}  % disable illegal ligatures
\fi
\usepackage[]{natbib}
\bibliographystyle{plainnat}
\IfFileExists{bookmark.sty}{\usepackage{bookmark}}{\usepackage{hyperref}}
\IfFileExists{xurl.sty}{\usepackage{xurl}}{} % add URL line breaks if available
\urlstyle{same}
\hypersetup{
  pdftitle={Stat 230 Winter 2024},
  pdfauthor={Chi-Kuang Yeh},
  hidelinks,
  pdfcreator={LaTeX via pandoc}}

\title{Stat 230 Winter 2024}
\author{Chi-Kuang Yeh}
\date{2024-01-10}

\usepackage{amsthm}
\newtheorem{theorem}{Theorem}[chapter]
\newtheorem{lemma}{Lemma}[chapter]
\newtheorem{corollary}{Corollary}[chapter]
\newtheorem{proposition}{Proposition}[chapter]
\newtheorem{conjecture}{Conjecture}[chapter]
\theoremstyle{definition}
\newtheorem{definition}{Definition}[chapter]
\theoremstyle{definition}
\newtheorem{example}{Example}[chapter]
\theoremstyle{definition}
\newtheorem{exercise}{Exercise}[chapter]
\theoremstyle{definition}
\newtheorem{hypothesis}{Hypothesis}[chapter]
\theoremstyle{remark}
\newtheorem*{remark}{Remark}
\newtheorem*{solution}{Solution}
\begin{document}
\maketitle

{
\setcounter{tocdepth}{1}
\tableofcontents
}
The course information is given below

\hypertarget{information-of-the-course}{%
\chapter{Information of the course}\label{information-of-the-course}}

The purpose of this page is to hold some of the additional materials provided by myself. Students should consult UW \href{https://api-4ccc589b.duosecurity.com/frame/v4/preauth/healthcheck?sid=frameless-c0657e9d-cb86-4ac9-a6a7-fd054ae21fd5}{Learn} system.

\hypertarget{course-description}{%
\section{Course description}\label{course-description}}

This course provides an introduction to probability models including sample spaces, mutually exclusive and independent events, conditional probability and Bayes' Theorem. The named distributions (Discrete Uniform, Hypergeometric, Binomial, Negative Binomial, Geometric, Poisson, Continuous Uniform, Exponential, Normal (Gaussian), and Multinomial) are used to model real phenomena. Discrete and continuous univariate random variables and their distributions are discussed. Joint probability functions, marginal probability functions, and conditional probability functions of two or more discrete random variables and functions of random variables are also discussed. Students learn how to calculate and interpret means, variances and covariances particularly for the named distributions. The Central Limit Theorem is used to approximate probabilities.

\hypertarget{instructor}{%
\subsection{Instructor}\label{instructor}}

Chi-Kuang Yeh, I am a postdoc at the \emph{Department of Statistics and Actuarial Science}.

\begin{itemize}
\tightlist
\item
  Office: M3--3102 Desk 10. I will hold office hour in another location.
\item
  Email: \href{mailto:chi-kuang.yeh@uwaterloo.ca}{\nolinkurl{chi-kuang.yeh@uwaterloo.ca}}
\end{itemize}

\hypertarget{course-coordinator}{%
\subsection{Course Coordinator}\label{course-coordinator}}

Dr.~\href{https://uwaterloo.ca/scholar/ehintz}{Erik Hintz}.

\begin{itemize}
\tightlist
\item
  Office: M3--2106
\item
  Email: \href{mailto:erik.hintz@uwaterloo.ca}{erik.hintz@uuwaterloo.ca}
\end{itemize}

\hypertarget{logistic-issue}{%
\subsection{Logistic Issue}\label{logistic-issue}}

Contact Divya Lala

\begin{itemize}
\tightlist
\item
  Email: \href{mailto:divya.lala@uwaterloo.ca}{\nolinkurl{divya.lala@uwaterloo.ca}} or the undergrad advising email \href{mailto:sasugradadv@uwaterloo.ca}{\nolinkurl{sasugradadv@uwaterloo.ca}}.
\end{itemize}

\begin{Shaded}
\begin{Highlighting}[]
\NormalTok{bookdown}\SpecialCharTok{::}\FunctionTok{serve\_book}\NormalTok{()}
\end{Highlighting}
\end{Shaded}

\hypertarget{lecture-1-january-08-2024}{%
\chapter{Lecture 1, January 08, 2024}\label{lecture-1-january-08-2024}}

In this lecture, we went over
1. Course syllabus and rules
2. Chapter 1 -- Basic definition of probability. We also saw the potential ambiguities when defining probabilities.

\hypertarget{lecture-2-january-10-2024}{%
\chapter{Lecture 2, January 10, 2024}\label{lecture-2-january-10-2024}}

In this lecture, we went over some basic definitions from the set theory, and using them as the building block for the rest of the course. We started Chapter 2 today, with many definitions.

As for the set operations, \(\cup,\cap,A^c,...\), the Venn diagrams help to visual the meaning behind. Here is a good reference \href{https://www.edrawmax.com/article/venn-diagram-symbols-and-set-notations.html}{HERE}.

\begin{definition}[sample space]
A \textbf{sample space} \(S\) is a \textbf{set} of distinct outcomes of an experiment with the property that in a single trial of the experiment only one of these outcomes occurs.
\end{definition}

\begin{definition}[Discrete and non-discrete sample space]
A sample space \(S\) is said to be \textbf{discrete} if it is finite, or ``countably infinite'' (i.e.,there is a one-to-one correspondence with the natural numbers). Otherwise a sample space is said to be \textbf{non-discrete}.
\end{definition}

\begin{definition}[Event]
An \textbf{event} is a subset of the sample space that can be assigned probability.
\end{definition}

\begin{definition}[Simple/Compound event]
Let \(S\) be discrete and \(A\subset S\) an event. If \(A\) is indivisible so it contains only one point, we call it a \textbf{simple event}, otherwise \textbf{compound event}.
\end{definition}

\begin{definition}[Probability distribution]
Let \(S=\{a_1,a_2,\dots\}\) be discrete. Assign numbers \(P(\{a_i\})\) (or short: \(P(a_i)\)), \(i=1,2,\dots\), so that

1.\(0\leq P(a_i)\leq 1,\quad i=1,2,\dots\)
2. \(\sum_{\text{all }i}P(a_i)=1\).

We then call the set of probabilities \(\{P(a_i):i=1,2,\dots\}\) a \textbf{probability distribution}.
\end{definition}

\begin{definition}
Let \(S=\{a_1,a_2,\dots\}\) discrete. From any prob. distribution \(P\) on \(S\) we can define a prob. measure on \$ \{\mathcal S\} = 2\^{}S\$ (set of all subsets of \(S\)) by
\[\forall A \subseteq S \qquad P(A)=\sum_{a_i\in A}P(a_i).\]
\end{definition}

\begin{definition}[Equally likely]
We say a sample space \(S\) with a finite number of outcomes is \textbf{equally likely} if the probability of every individual outcome in \(S\) is the same.
\end{definition}

  \bibliography{book.bib,packages.bib}

\end{document}
