% Options for packages loaded elsewhere
\PassOptionsToPackage{unicode}{hyperref}
\PassOptionsToPackage{hyphens}{url}
%
\documentclass[
]{book}
\usepackage{amsmath,amssymb}
\usepackage{iftex}
\ifPDFTeX
  \usepackage[T1]{fontenc}
  \usepackage[utf8]{inputenc}
  \usepackage{textcomp} % provide euro and other symbols
\else % if luatex or xetex
  \usepackage{unicode-math} % this also loads fontspec
  \defaultfontfeatures{Scale=MatchLowercase}
  \defaultfontfeatures[\rmfamily]{Ligatures=TeX,Scale=1}
\fi
\usepackage{lmodern}
\ifPDFTeX\else
  % xetex/luatex font selection
\fi
% Use upquote if available, for straight quotes in verbatim environments
\IfFileExists{upquote.sty}{\usepackage{upquote}}{}
\IfFileExists{microtype.sty}{% use microtype if available
  \usepackage[]{microtype}
  \UseMicrotypeSet[protrusion]{basicmath} % disable protrusion for tt fonts
}{}
\makeatletter
\@ifundefined{KOMAClassName}{% if non-KOMA class
  \IfFileExists{parskip.sty}{%
    \usepackage{parskip}
  }{% else
    \setlength{\parindent}{0pt}
    \setlength{\parskip}{6pt plus 2pt minus 1pt}}
}{% if KOMA class
  \KOMAoptions{parskip=half}}
\makeatother
\usepackage{xcolor}
\usepackage{longtable,booktabs,array}
\usepackage{calc} % for calculating minipage widths
% Correct order of tables after \paragraph or \subparagraph
\usepackage{etoolbox}
\makeatletter
\patchcmd\longtable{\par}{\if@noskipsec\mbox{}\fi\par}{}{}
\makeatother
% Allow footnotes in longtable head/foot
\IfFileExists{footnotehyper.sty}{\usepackage{footnotehyper}}{\usepackage{footnote}}
\makesavenoteenv{longtable}
\usepackage{graphicx}
\makeatletter
\def\maxwidth{\ifdim\Gin@nat@width>\linewidth\linewidth\else\Gin@nat@width\fi}
\def\maxheight{\ifdim\Gin@nat@height>\textheight\textheight\else\Gin@nat@height\fi}
\makeatother
% Scale images if necessary, so that they will not overflow the page
% margins by default, and it is still possible to overwrite the defaults
% using explicit options in \includegraphics[width, height, ...]{}
\setkeys{Gin}{width=\maxwidth,height=\maxheight,keepaspectratio}
% Set default figure placement to htbp
\makeatletter
\def\fps@figure{htbp}
\makeatother
\setlength{\emergencystretch}{3em} % prevent overfull lines
\providecommand{\tightlist}{%
  \setlength{\itemsep}{0pt}\setlength{\parskip}{0pt}}
\setcounter{secnumdepth}{5}
\usepackage{booktabs}
\usepackage{amsmath}
\usepackage{mathtools}
\ifLuaTeX
  \usepackage{selnolig}  % disable illegal ligatures
\fi
\usepackage[]{natbib}
\bibliographystyle{plainnat}
\IfFileExists{bookmark.sty}{\usepackage{bookmark}}{\usepackage{hyperref}}
\IfFileExists{xurl.sty}{\usepackage{xurl}}{} % add URL line breaks if available
\urlstyle{same}
\hypersetup{
  pdftitle={Stat 230 Introduction to Probability},
  hidelinks,
  pdfcreator={LaTeX via pandoc}}

\title{Stat 230 Introduction to Probability}
\usepackage{etoolbox}
\makeatletter
\providecommand{\subtitle}[1]{% add subtitle to \maketitle
  \apptocmd{\@title}{\par {\large #1 \par}}{}{}
}
\makeatother
\subtitle{Winter 2024}
\author{Chi-Kuang Yeh\\
University of Waterloo}
\date{2024-01-27}

\usepackage{amsthm}
\newtheorem{theorem}{Theorem}[chapter]
\newtheorem{lemma}{Lemma}[chapter]
\newtheorem{corollary}{Corollary}[chapter]
\newtheorem{proposition}{Proposition}[chapter]
\newtheorem{conjecture}{Conjecture}[chapter]
\theoremstyle{definition}
\newtheorem{definition}{Definition}[chapter]
\theoremstyle{definition}
\newtheorem{example}{Example}[chapter]
\theoremstyle{definition}
\newtheorem{exercise}{Exercise}[chapter]
\theoremstyle{definition}
\newtheorem{hypothesis}{Hypothesis}[chapter]
\theoremstyle{remark}
\newtheorem*{remark}{Remark}
\newtheorem*{solution}{Solution}
\begin{document}
\maketitle

{
\setcounter{tocdepth}{1}
\tableofcontents
}
\hypertarget{information-of-the-course}{%
\chapter{Information of the course}\label{information-of-the-course}}

The purpose of this page is to hold some of the additional materials provided by myself. Students should consult UW \href{https://api-4ccc589b.duosecurity.com/frame/v4/preauth/healthcheck?sid=frameless-c0657e9d-cb86-4ac9-a6a7-fd054ae21fd5}{Learn} system.

\hypertarget{course-description}{%
\section{Course description}\label{course-description}}

This course provides an introduction to probability models including sample spaces, mutually exclusive and independent events, conditional probability and Bayes' Theorem. The named distributions (Discrete Uniform, Hypergeometric, Binomial, Negative Binomial, Geometric, Poisson, Continuous Uniform, Exponential, Normal (Gaussian), and Multinomial) are used to model real phenomena. Discrete and continuous univariate random variables and their distributions are discussed. Joint probability functions, marginal probability functions, and conditional probability functions of two or more discrete random variables and functions of random variables are also discussed. Students learn how to calculate and interpret means, variances and covariances particularly for the named distributions. The Central Limit Theorem is used to approximate probabilities.

\hypertarget{instructor}{%
\subsection{Instructor}\label{instructor}}

Chi-Kuang Yeh, I am a postdoc at the \emph{Department of Statistics and Actuarial Science}.

\begin{itemize}
\tightlist
\item
  Office: M3--3102 Desk 10. I will hold office hour in another location.
\item
  Email: \href{mailto:chi-kuang.yeh@uwaterloo.ca}{\nolinkurl{chi-kuang.yeh@uwaterloo.ca}}
\end{itemize}

\hypertarget{course-coordinator}{%
\subsection{Course Coordinator}\label{course-coordinator}}

Dr.~\href{https://uwaterloo.ca/scholar/ehintz}{Erik Hintz}.

\begin{itemize}
\tightlist
\item
  Office: M3--2106
\item
  Email: \href{mailto:erik.hintz@uwaterloo.ca}{erik.hintz@uuwaterloo.ca}
\end{itemize}

\hypertarget{logistic-issue}{%
\subsection{Logistic Issue}\label{logistic-issue}}

Contact Divya Lala

\begin{itemize}
\tightlist
\item
  Email: \href{mailto:divya.lala@uwaterloo.ca}{\nolinkurl{divya.lala@uwaterloo.ca}} or the undergrad advising email \href{mailto:sasugradadv@uwaterloo.ca}{\nolinkurl{sasugradadv@uwaterloo.ca}}.
\end{itemize}

\hypertarget{exam-and-tutorial-assessment-date}{%
\subsection{EXAM and Tutorial assessment Date}\label{exam-and-tutorial-assessment-date}}

Midterm

\begin{itemize}
\tightlist
\item
  Midterm 1: February 08, 16:30--17:50
\item
  Midterm 2: March 14, 16:30--17:50
\end{itemize}

Final

\begin{itemize}
\tightlist
\item
  To be announced by the university
\end{itemize}

Tutorial assessment

\begin{itemize}
\tightlist
\item
  Tutorial quiz 1: January 26
\item
  Tutorial test 1: February 02
\item
  Tutorial quiz 2: March 01
\item
  Tutorial test 2: March 08
\item
  Tutorial quiz 3: March 22
\item
  Tutorial test 3: April 05
\end{itemize}

\hypertarget{chapters-and-associated-lectures}{%
\section{Chapters and associated Lectures}\label{chapters-and-associated-lectures}}

Those chapters are based on the lecture notes. The lecture covered is based on \emph{Section 002}. This part will be updated frequently.

\begin{longtable}[]{@{}lll@{}}
\toprule\noalign{}
Chapter & Title & Lecture Covered \\
\midrule\noalign{}
\endhead
\bottomrule\noalign{}
\endlastfoot
1 & Introduction to Probability & 1 \\
2 & Mathematical Probability Models & 2--3 \\
3 & Probability and Counting Techniques & 3--6 \\
4 & Probability rules and Conditional Probability & 6-- \\
5 & TBA & TBA \\
6 & TBA & TBA \\
7 & TBA & TBA \\
8 & TBA & TBA \\
9 & TBA & TBA \\
10 & TBA & TBA \\
\end{longtable}

\hypertarget{lecture-1-january-08-2024}{%
\chapter{Lecture 1, January 08, 2024}\label{lecture-1-january-08-2024}}

In this lecture, we went over

\begin{enumerate}
\def\labelenumi{\arabic{enumi}.}
\tightlist
\item
  Course syllabus and rules
\item
  Chapter 1 -- Basic definition of probability. We also saw the potential ambiguities when defining probabilities.
\end{enumerate}

\begin{center}\rule{0.5\linewidth}{0.5pt}\end{center}

\begin{definition}[Classical Definition of probability]
The \textbf{classical} definition: The probability of some event is
\[
\frac{\mathrm{number~of~ways~the~event~can~occur~}}
{\mathrm{{the~total~number~of~possible~outcomes}}},
\]
provided all outcomes are \emph{equally likely}.
\end{definition}

\begin{center}\rule{0.5\linewidth}{0.5pt}\end{center}

\begin{definition}[Relative Frequency Definition of of probability]
The \textbf{relative frequency} definition: The probability of an event
is the (limiting) proportion (or fraction) of times the event occurs in a very
long series of repetitions of an experiment.
\end{definition}

\begin{center}\rule{0.5\linewidth}{0.5pt}\end{center}

\begin{definition}[Subjective Definition of Probability]
The \textbf{subjective} definition: The probability of an event is a measure of how sure the person making the statement is that the event will happen.
\end{definition}

\begin{center}\rule{0.5\linewidth}{0.5pt}\end{center}

\textbf{Problem}: Each of the above definitions has pitfall:

\begin{itemize}
\tightlist
\item
  Classical: We may not be able to know the total number of possible outcomes, or it may be uncountable
\item
  Relative frequency: We need ``repetition'', which is often expensive and may not be possible.
\item
  Subjective: We want the probability to be consistent across different people, and and be rigorously defined.
\end{itemize}

\hypertarget{lecture-2-january-10-2024}{%
\chapter{Lecture 2, January 10, 2024}\label{lecture-2-january-10-2024}}

In this lecture, we went over some basic definitions from the set theory, and using them as the building block for the rest of the course. We started Chapter 2 today, with many definitions.

As for the set operations, \(\cup,\cap,A^c,...\), the Venn diagrams help to visual the meaning behind. Here is a good reference \href{https://www.edrawmax.com/article/venn-diagram-symbols-and-set-notations.html}{HERE}.

\begin{definition}[sample space]
A \textbf{sample space} \(S\) is a \emph{set} of distinct outcomes of an experiment with the property that in a single trial of the experiment only one of these outcomes occurs.
\end{definition}

\begin{definition}[Discrete and non-discrete sample space]
A sample space \(S\) is said to be \textbf{discrete} if it is finite, or ``countably infinite'' (i.e.,there is a one-to-one correspondence with the natural numbers). Otherwise a sample space is said to be \textbf{non-discrete}.
\end{definition}

\begin{definition}[Event]
An \textbf{event} is a subset of the sample space that can be assigned probability.
\end{definition}

\begin{definition}[Simple and Compound event]
Let \(S\) be discrete and \(A\subset S\) an event. If \(A\) is indivisible so it contains only one point, we call it a \textbf{simple event}, otherwise \textbf{compound event}.
\end{definition}

\begin{definition}[Probability distribution]
Let \(S=\{a_1,a_2,\dots\}\) be discrete. Assign numbers \(P(\{a_i\})\) (or short: \(P(a_i)\)), \(i=1,2,\dots\), so that

1.\(0\leq P(a_i)\leq 1,\quad i=1,2,\dots\)
2. \(\sum_{\text{all }i}P(a_i)=1\).

We then call the set of probabilities \(\{P(a_i):i=1,2,\dots\}\) a \textbf{probability distribution}.
\end{definition}

\begin{definition}
Let \(S=\{a_1,a_2,\dots\}\) discrete. From any prob. distribution \(P\) on \(S\) we can define a prob. measure on \$ \{\mathcal S\} = 2\^{}S\$ (set of all subsets of \(S\)) by
\[\forall A \subseteq S \qquad P(A)=\sum_{a_i\in A}P(a_i).\]
\end{definition}

\begin{definition}[Equally likely]
We say a sample space \(S\) with a finite number of outcomes is \textbf{equally likely} if the probability of every individual outcome in \(S\) is the same.
\end{definition}

Observe that

\begin{itemize}
\item
  If \(|A|\) denote the number of outcomes in an event \(A\). In case of an equally likely sample space,
  \[
  1=P(S)=\sum_{i=1}^{|S|}P(a_{i})= P(a_i)|S|.
  \]
  \[
  P(a_i)=\frac{1}{|S|}.
  \]
\item
  Hence,
  \[P(A) = \sum_{i:\;a_i \in A} P(a_i) = \sum_{i:\;a_i \in A} \frac{1}{|S|} =|A|\cdot  \frac{1}{|S|}\]
\end{itemize}

\textbf{Conclusion}: In a \textbf{finite, equally likely sample space}, the probability of an event \(A\) can be computed as
\[
P(A) = \sum_{i:\;a_i \in A} P(a_i) = \frac{|A|}{|S|}.
\]

\hypertarget{questions-from-the-class}{%
\section{Questions from the class}\label{questions-from-the-class}}

\begin{enumerate}
\def\labelenumi{\arabic{enumi}.}
\tightlist
\item
  What is the difference between ``countably infinite'' v.s. ``infinite''?
\end{enumerate}

Ans: A set is \emph{countably infinite} if its elements can be put in one-to-one correspondence with the set of natural numbers \(\mathbb{N}\). Alternatively, you can think that a set is countably infinite if you can count off all elements in the set in such a way that, even though the counting will take forever, you will get to any particular element in a finite amount of time. {[}\href{https://mathinsight.org/definition/countably_infinite}{A good reference page to read}{]}. If a set is not countable or countably infinite, it is infinite.

\begin{enumerate}
\def\labelenumi{\arabic{enumi}.}
\setcounter{enumi}{1}
\tightlist
\item
  Why do we have something such as \(2^\mathcal{S}\) in the lecture?
\end{enumerate}

Ans: It is related to something called the \emph{power set}. The power set consists all the possible subset of a set \(\mathcal{S}\). In a subset of \(S\), (i.e.~\(A \subseteq \mathcal{S}\), every element in \(\mathcal{S}\) may be either in \(A\) or not in \(A\). Which means, each element has two possibilities, in \(A\) or not in \(A\). Hence, the cardinality (i.e.~the size) of the power set is \(2^\mathcal{S}\).

\begin{enumerate}
\def\labelenumi{\arabic{enumi}.}
\setcounter{enumi}{2}
\tightlist
\item
  What did we mean by ``order does not matter'' and ``order matters'' during the lecture.
\end{enumerate}

\begin{itemize}
\item
  Order does not matter: I said when you write out the element of a set, the order does not matter. For instance, in the rolling a six-sided dice, which side would be faced on the top example, we can write \(\mathcal{S} = \{1,2,3,4,5,6\}\), or \(\mathcal{S}^\prime=\{6,5,4,3,2,1\}\), and those two sets are essentially equal to each other. To represent a set, the order does not matter, but we tend to write in a way that is intuitive and easy to understand.
\item
  Order does matter: In rolling two dices example, the dots show on each of the dice is an \emph{ordered pair}, denoted by \((x,y)\). Hence, for instance, \((1,2)\) and \((2,1)\) are different. It is problem-dependent so be careful.
\end{itemize}

\hypertarget{lecture-3-january-12-2024}{%
\chapter{Lecture 3, January 12, 2024}\label{lecture-3-january-12-2024}}

\begin{definition}[Odds]
Odds \textbf{in favour} of an event \(A\) occurring is
\[
  O(A) := \frac{P(A)}{1-P(A)}.
\]
Odds again an event \(A\) is
\[
  \frac{1-P(A)}{P(A)}.
\]
\end{definition}

The range of the odds is \([0,\infty)\).

It provide a measure of the likelihood of a particular outcome to happen.

Abbreviation: ````p:q''.

Note: Probability may be defined through the odds as follow.
\begin{align*}
  &O(A) := \frac{P(A)}{1-P(A)} \\
  &\implies O(A) - P(A)O(A) = P(A) \\
  &\implies O(A) = P(A) (1+O(A)) \\
  &\implies P(A) = \frac{O(A)} {1+O(A)}
\end{align*}

Note: In finite, equally likely sample spaces, computing probabilities amounts to \emph{counting the number of elements in a set}. It will often be difficult to do this manually, so we are looking for clever \emph{counting techniques} in the next chapter.

\begin{center}\rule{0.5\linewidth}{0.5pt}\end{center}

\textbf{Chapter 3 Counting Techniques}

Addition rule v.s. Multiplication rule

For addition rule

\begin{itemize}
\tightlist
\item
  Keyword for addition rule is ``\textbf{OR}'';
\item
  \(|A|\) is defined to be the size of the set, aka the cardinality of the set.
\item
  If \(A\) and \(B\) are \emph{disjoint} (i.e.~\(A\cap B = \emptyset\)), then \(|A\cup B| = |A| + |B|\).
\item
  \(A\cup A^c = S\) where \(A \cap A^c = \emptyset\). Thus \(|S|=|A|+|A^c|\).
\end{itemize}

For multiplication rule

\begin{itemize}
\tightlist
\item
  for multiplication rule is ``\textbf{AND}''
\item
  An ordered k-tuple is an ordered set of \(k\) values: \((a_1,a_2,\dots,a_k)\). If the outcomes in A can be wrttien as an ordered k-tuple where there are \(n_1\) choices for \(a_1\), \(n_2\) choices for \(a_2,\dots\) and in general \(n_i\) choices for \(a_i\), then
  \[
  |A| = n_1n_2\cdots n_k = \prod_{i=1}^k n_i.
    \]
\end{itemize}

\begin{center}\rule{0.5\linewidth}{0.5pt}\end{center}

\begin{definition}[Factorial]
Given \(n\) distinct objects, there are
\[
n! = n \times (n-1) \times \ldots 2 \times 1,
\]
different ordered arrangements of length \(n\) that can be made. Note that, we define, \(0! = 1\).
\end{definition}

\begin{itemize}
\tightlist
\item
  We pronounce \(n!\) as ``n factorial''.
\item
  The following recursive definition is useful:
  \[ 
  n! = n \cdot (n-1)!
  \]

  \item

  When working with factorials, we can often cancel terms, e.g.,
  \[ \frac{9!}{7!} = \frac{9\cdot 8 \cdot 7\cdot 6 \cdot \dots \cdot 2 \cdot 1}{7\cdot 6 \cdot \dots \cdot 2 \cdot 1}=9\cdot 8 = 72\]
\end{itemize}

\begin{center}\rule{0.5\linewidth}{0.5pt}\end{center}

\begin{definition}[Permutation]
Given \(n\) distinct objects, a \textbf{permutation} of size \(k\) is an \(ordered\) subset of \(k\) of the individuals. The number of permutations of size \(k\) taken from \(n\) objects is denoted \(n^{(k)}\) and
\[
n^{(k)}=n(n-1)\dots (n-k+1) =\frac{n!}{(n-k)!}.
\]
\end{definition}

The tricky part of this definition is the word ``ordered''. An ordering need not be numerical, for example assigning labels like ``President'' and ``Vice-President'' has the effect of ordering the individuals.

\hypertarget{questions-from-the-class-1}{%
\section{Questions from the class}\label{questions-from-the-class-1}}

\begin{enumerate}
\def\labelenumi{\arabic{enumi}.}
\tightlist
\item
  Can we express the odds as \(a:b\)?
\end{enumerate}

YES. For instance, the example we saw in class (or the clicker question 1), if we roll a fair six sided dice, and let our event \(A:=\{\text{# is } 5 \}\). Then the odds \(O(A)=\frac{1/6}{1/5}=\frac{1}{5}\). We can see that, there is exactly one possibility we have event \(A\), whereas there are 5 possibilities that \(A\) does not happen (i.e.~the number we roll out is \(1,2,3,4,6\)). We \textbf{can} abbreviate it as ``1:5''. For a good example of Odds, \href{https://en.wikipedia.org/wiki/Odds}{WIKI} provides a good one.

\hypertarget{lecture-4-january-15-2024}{%
\chapter{Lecture 4, January 15, 2024}\label{lecture-4-january-15-2024}}

\begin{definition}[Combination]
Given \(n\) distinct objects, a \emph{combination} of size \(k\) is an \emph{unordered} subset of \(k\) of the individuals. The number of combinations of size \(k\) taken from \(n\) objects is denoted \({n \choose k}\) or \({}_n C_k\) and can be computed as
\[
{n \choose k}=\frac{n^{(k)}}{k!}=\frac{n!}{(n-k)!\ k!}.
\]
\end{definition}

\hypertarget{lecture-5-january-17-2024}{%
\chapter{Lecture 5, January 17, 2024}\label{lecture-5-january-17-2024}}

Properties of the Binomial coefficients

There are some useful/important results about permutation and combination.

\begin{enumerate}
\def\labelenumi{\arabic{enumi}.}
\tightlist
\item
  \(n^{(k)} = n (n - 1)^{(k-1)}\) for \(k \geq 1\)
\item
  \({n \choose k} = \frac{n^{(k)}}{k!}\)
\item
  \({n \choose k} = {n \choose n-k}\) for \(k \geq 0\)
\item
  \({n \choose k} = {n-1 \choose k-1} + {n-1 \choose k}\)
\item
  Binomial theorem: \((1 + x)^n = \sum_{k=0}^n {n \choose k} x^k\)
\item
  \({n \choose k}\) is equal to the \(k\)th entry in the \(n\)th row of \textbf{Pascal's triangle}.
\end{enumerate}

Note: Many of these idenetity may be proven using something called \emph{combinatorial proof}. See \href{https://en.wikipedia.org/wiki/Combinatorial_proof}{Wiki} for an (easy) example.

\begin{proof}[4]
\begin{align*}
{n-1 \choose k-1} + {n-1 \choose k} &= \frac{(n-1)!}{(k-1)! (n-k)!} + \frac{(n-1)!}{k! (n-k-1)!}\\
&= \frac{(n-1)!k }{(k-1)! (n-k)!k} + \frac{(n-1)!(n-k)}{k! (n-k-1)!(n-1)} \\
&= \frac{(n-1)!k + (n-1)! (n-k)}{k! (n-k)!} \\
&- \frac{(n-1)!( k + (n-k))}{k! (n-k)!} \\
&= \frac{(n-1)! n}{k! (n-k)!} \\
&= {n \choose k}
\end{align*}
\end{proof}

\begin{center}\rule{0.5\linewidth}{0.5pt}\end{center}

Aside: Stirling's formula

\(n!\) grows really fast as \(n\) increases, so sometimes we need to approximate its value for computational reasons.

\textbf{Stirling's formula} provides one such method, and it is given by

\[
n! \sim \sqrt{2 \pi n} \left( \frac{n}{e} \right)^n,
\]
where \(\sim\) means their ratio approaches 1 as \(n\) goes to infinity.

We won't need this approximation, but it's useful to know it exists.

Example of use:
Show that \(2^{-2n}\binom{2n}{n} \approx \sqrt{\frac{2}{\pi n}}\)

\begin{center}\rule{0.5\linewidth}{0.5pt}\end{center}

\hypertarget{example-in-class}{%
\section{Example in class}\label{example-in-class}}

\begin{example}[Application of Stirling's Formula/Approximation for factorial]
Show that \(2^{-2n}{2n \choose n} \approx \sqrt{\frac{1}{\pi n}}\)

\begin{align*}
  2^{-2n}{2n \choose n} &= 2^{-2n}\frac{2n!}{n!n!} \\
  &\approx 2^{-2n} \frac{\sqrt{2\pi (2n)} (2n/e)^{2n}}{\sqrt{2\pi n} (n/e)^{n}\sqrt{2\pi n} (n/e)^{n}}\\
  &= 2^{-2n} \frac{\sqrt{4}}{\sqrt{2}\sqrt{2}} \frac{\sqrt{\pi n}}{\sqrt{\pi n}\sqrt{\pi n}} \frac{(2n)^{2n}}{n^n n^n} \frac{e^{-2n}}{e^{-2n}}\\
  &=  2^{-2n} \frac{1}{\sqrt{\pi n}} 2^{2n}\\
  &= \frac{1}{\sqrt{\pi n}} = \sqrt{\frac{1}{\pi n}}
\end{align*}
\end{example}

\hypertarget{lecture-6-january-19-2024}{%
\chapter{Lecture 6, January 19, 2024}\label{lecture-6-january-19-2024}}

\hypertarget{multinomial-coefficient}{%
\subsection{Multinomial Coefficient}\label{multinomial-coefficient}}

\begin{definition}[Multinomial Coefficient]
Consider \(n\) objects which consist of \(k\) types. Suppose that there are \(n_1\) objects which are of type 1, \(n_2\) which are of type 2, and in general \(n_i\) objects of type \(i\). Then there are
\[
\frac{n!}{n_1 ! n_2! \dots n_k !}
\]
distinguishable arrangements of the \(n\) objects. This quantity is known as a \textbf{multinomial coefficient} and denoted by
\[
\binom{n}{n_1,n_2,\dots,n_k}= \frac{n !}{n_1 ! n_2! \dots n_k !}.
\]
\end{definition}

\textbf{Note}: Multinomial coefficient is an extension of the \emph{binomial coefficient}. In binomial coefficient, there are only \textbf{two groups/objects}, and the first type has size \(n_1\) and the size of the second type is consequently \(n-n_1\), where \(n\) is the total number of objects. Hence we have \({n \choose n_1} = \frac{n!}{n_1! (n-n_1)!}\). Try to compare this with the multinomial coefficient.

\begin{center}\rule{0.5\linewidth}{0.5pt}\end{center}

\hypertarget{the-birthday-problem}{%
\subsection{The Birthday Problem}\label{the-birthday-problem}}

Suppose a room contains \(n\) people. What is the probability at least two people in the room share a birthday?

\textbf{Assumption}: Suppose that each of the \(n\) people is equally likely to have any of the 365 days of the year as their birthday, so that all possible combinations of birthdays are equally likely.

Let \(A\) be the event that at least two people share a birthday. Then
\[ P(A) = 1 - P(A^c),\]
where \(A^c\) is the event that nobody shares birthday with each other.

For \(n\) people to have unique birthdays, we need to arrange them among 365 days w/o replacement. Thus,
\[|A^c| = 365^{(n)}.\]

For the size of the sample space, we see that each person has 365 possibilities for their birthday. Thus,
\[|S| = 365^n.\]

Since we are assuming that all possible combinations of birthdays are equally likely, our desired probability becomes
\[
P(A) = 1 - P(A^c) = 1 - \frac{365^{(n)}}{365^n} = 1 - \frac{n! {365 \choose n}}{365^n}.
\]

For \(n\in\{100, 30, 23\}\) we find
\[P(A_{100})= .9999997,\;\;\; P(A_{30})=.7063 \;\;\;\; P(A_{23})=.5073.\]

\begin{center}\rule{0.5\linewidth}{0.5pt}\end{center}

\hypertarget{chapter-4-probbility-rules-and-conditional-probability}{%
\subsection{Chapter 4 Probbility Rules and Conditional Probability}\label{chapter-4-probbility-rules-and-conditional-probability}}

Review the Venn Diagram

\hypertarget{lecture-7-january-22-2024}{%
\chapter{Lecture 7, January 22, 2024}\label{lecture-7-january-22-2024}}

\hypertarget{some-terminology-about-the-set-thoery.}{%
\subsection{Some terminology about the set thoery.}\label{some-terminology-about-the-set-thoery.}}

\hypertarget{fundamental-law-of-set-algebra}{%
\subsubsection{Fundamental law of set algebra}\label{fundamental-law-of-set-algebra}}

Let \(A\) and \(B\) be any arbitrary sets/events.

\begin{enumerate}
\def\labelenumi{\arabic{enumi}.}
\tightlist
\item
  Commutative
\end{enumerate}

\[
  A\cup B = B\cup A \quad \text{ and } A\cap B = B\cap A.
\]

\begin{enumerate}
\def\labelenumi{\arabic{enumi}.}
\setcounter{enumi}{1}
\tightlist
\item
  Associativity
\end{enumerate}

\[
  (A\cup B)\cup C = A \cup (B\cup C), \quad \text{and } (A\cap B)\cap C =  A \cap (B \cap C).
\]

\begin{enumerate}
\def\labelenumi{\arabic{enumi}.}
\setcounter{enumi}{2}
\tightlist
\item
  Distributive Law
\end{enumerate}

\[
  A\cup (B\cap C) = (A \cup B) \cap (A \cap  C) , \quad \text{ and } A \cap (B\cup C) =  (A\cap B) \cup (A\cap C)
\]

\hypertarget{demorgans-laws}{%
\subsubsection{DeMorgan's Laws}\label{demorgans-laws}}

\begin{enumerate}
\def\labelenumi{\arabic{enumi}.}
\item
  \((A\cup B)^c = A^c \cap B^c\) (Complement of an union is the intersection of the complements)
\item
  \(A\cap B)^c = A^c \cup B^c\) (Complement of an intersection is the union of the complements)
\end{enumerate}

\hypertarget{inclusion-exclusion-principle}{%
\subsubsection{Inclusion Exclusion Principle}\label{inclusion-exclusion-principle}}

\begin{enumerate}
\def\labelenumi{\arabic{enumi}.}
\item
  \(P(A\cup B ) = P(A) + P(B) - P(A\cap B)\)
\item
  \(P(A\cup B \cup C) = P(A) + P(B) + P(C) - P(A\cap B) - P(A \cap C) - P(B \cap C) + P(A\cap B \ cap C)\)
\item
  Note that we can generalized the (2), and obtain the following by \emph{inducation.} For arbitrary events \(A_1,A_2,\cdots,A_n,\quad n\ge 2\),
  \begin{align*}
  P(\bigcup_{i=1}^n A_i)  &  =\sum_{i}P(A_{i}%
  )-\sum_{i<j}P(A_{i}A_{j})+\sum_{i<j<k}P(A_{i}A_{j}A_{k})\\
  &  -\sum_{i<j<k<l}P(A_{i}A_{j}A_{k}A_{l})+\cdots
  \end{align*}
\end{enumerate}

\hypertarget{independence}{%
\subsection{Independence}\label{independence}}

\begin{definition}[independence]
Any two events \(A\) and \(B\) are said to be \textbf{independent} if
\[
  P(A \cap B) = P(A)\times P(B).
\]
\end{definition}

Note: Intuitively, it means that two events do not have any influence of each other. You will see that how this concept plays an important role in statistics, in particular through something called the \emph{covariance}, which is beyond this course so do not worry about this for now.

\hypertarget{independence-v.s.-multually-exclusivedisjoint}{%
\subsection{Independence v.s. Multually Exclusive/Disjoint}\label{independence-v.s.-multually-exclusivedisjoint}}

Recall the definition of mutually exclusive

\begin{definition}[Mutually Exclusive]
Any two events \(A\) and \(B\) are said to be \textbf{mutually exclusive} or \textbf{disjoint} if
\[
  P(A \cap B) = 0.
\]
\end{definition}

Note:

\begin{enumerate}
\def\labelenumi{\arabic{enumi}.}
\item
  If \(A\) and \(B\) are mutually exclusive, \(A\) and \(B\) may NOT be independent!
\item
  \(A\) and \(B\) CAN only be mutually exclusive and independent when either \(A\), \(A\), or both are the empty set \(\emptyset\).
\end{enumerate}

\begin{lemma}
Let two events \(A\) and \(B\) such that NOT both events are trivial events (empty set). If \(A\) and \(B\) are independent and mutually exclusive/disjoint, then either \(P(A) = 0\) or \(P(B) = 0\).
\end{lemma}

\hypertarget{lecture-8-january-24-2024}{%
\chapter{Lecture 8, January 24, 2024}\label{lecture-8-january-24-2024}}

\begin{definition}[Conditional Probability]
The conditional probability of an event \(A\) given an event \(B\), assuming \(P(B)>0\), is

\[
  P(A \mid B) = \frac{P(A\cap B)}{P(B)}.
\]
\end{definition}

\begin{definition}[Equivalent definition of independence]
Two events \(A\) and \(B\) are independent, if
\[
P(A|B)=P(A),
\]
provided \(P(B)>0\).
\end{definition}

\hypertarget{properties-of-conditional-probability}{%
\subsection{Properties of Conditional Probability}\label{properties-of-conditional-probability}}

\begin{enumerate}
\def\labelenumi{\arabic{enumi}.}
\tightlist
\item
  \(0 \le P(A \mid B) \le 1\).
\end{enumerate}

This follows from the fact that if \(A \subset B\) then \(P(A) \le P(B)\)

\begin{enumerate}
\def\labelenumi{\arabic{enumi}.}
\setcounter{enumi}{1}
\item
  \(P(A^c \mid B) = 1-P(A \mid B)\).
\item
  If \(A_1\) and \(A_2\) are disjoint (i.e.~\(P(A_1\cap A_2)=\emptyset\): \(P(A_1 \cup A_2 \mid B) = P(A_1 \mid B) + P(A_2 \mid B)\).
\item
  \(P(S \mid B)= 1 = P(B \mid B)\).
\end{enumerate}

\begin{definition}[Product rule]
For any events \(A\) and \(B\), we have
\[
P(A\cap B) = P(A\mid B) P(B)  = P(B \mid A) P(A).
\]
\end{definition}

\hypertarget{lecture-9-january-26-2024}{%
\chapter{Lecture 9, January 26, 2024}\label{lecture-9-january-26-2024}}

\begin{definition}[Partition]
A sequence of sets \(B_1,B_2,...,B_k\) are said to \textbf{partition} the sample space \(S\) if \(B_i \cap B_j = \emptyset\) for all \(i \ne j\), and \(\cup_{j=1}^k B_j = S\).
\end{definition}

\begin{theorem}[Law of Total Probability]
Suppose that \(B_1,B_2,...,B_k\) partition \(S\). Then for any event \(A\),
\[
P(A) = P(A | B_1) P(B_1) + P(A | B_2) P(B_2) + \cdots +P(A | B_k) P(B_k).
\]
\end{theorem}

Note: It is a simple usage of LTP such that
\[
  P(A) = P(A\cap B) + P(A \cap B^c),
\]
since \(B\) and \(B^c\) partition \(S\) (i.e.~\(B\cup B^c = S\) and \(B\cap B^c = \emptyset\))

\begin{center}\rule{0.5\linewidth}{0.5pt}\end{center}

\hypertarget{bayes-rule}{%
\subsubsection{Bayes Rule}\label{bayes-rule}}

If we can calculate the conditional profitability, then we may calculate the desire probability by using 1) LTP, 2) definition of conditional probability, and 3) property of the sets (through the Venn diagrams). However, sometimes we have to \emph{FLIP} the event and the conditioning event. This brings us to the \textbf{Bayes Theorem}.

\begin{theorem}[Bayes Theorem]
Suppose that \(B_1,B_2,...,B_k\) partition \(S\). Then for any event \(A\),
\[
P(A) = P(A | B_1) P(B_1) + P(A | B_2) P(B_2) + \cdots +P(A | B_k) P(B_k).
\]
\end{theorem}

This concludes Chapter 4!.

  \bibliography{book.bib,packages.bib}

\end{document}
